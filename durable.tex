	\section{Durable Communities}
\vspace{0.2cm}

A durable community is one that is designed to last for generations. The intent towards durability does create some limitations on the kinds of communities that you can create. 

By durable we mean an organization that is intended to have stability long after all of the original organizers have moved on or passed the organization down to the next generation. Members of such an organization should have guaranteed rights to access the fruits of the organization, but do not have the right to destroy or dismember the organization. 

\subsection{How does an organization have durability?}

An organization may have durability when organizers create structures that add the design features of a stable commons. We can see a breakdown of these design principals in section 2. 

In addition to the organization structure, when decisions are made through a consensus process is also an important feature. We can see a breakdown of the consensus process in section 4.

Durability is supported by members having trust in mutual aid with each other. We can see a process for boot-strapping trust in section 6.  

\subsection{How does an organization lack durability?}

When members have a shareholder model then there needs to be a cash-out process. Cash-outs can destabilize an organization. This is treating shared resources like an investment, instead of a commons. 

When an organization is transitioned from one generation to the next as ownership, there is a problem where the inheritors may not be interested in participation. Disinterested members might want to exit the organization, and get their personal value back out of it. Members exiting without a desire to maintain stability can cause destabilization. 

\subsection{How can commons be made durable?}

Commons, if designed well, and governed well, could be a much more stable structure than cooperative private ownership. In the first generation, both a commons and a cooperatively owned business have similar benefits and stability to the members. The generational turnover is very different. A commons will use the decision making process established to accept new members when a previous member leaves or passes away. Allowing the commons to decide who ought to be included in future operation, stabilizes the organization. 

\subsection{"7 Generation" Thinking}

As a principal, we think that actions taken by an individual or group should take into account the effect 7 generations down the line. We think that building structures that provide abundant access to resources for generations from now is a design goal of organizing new communities and resources. 

