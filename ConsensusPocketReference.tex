\documentclass{article}
\usepackage{geometry}
\usepackage{eso-pic} % allows background placement
\geometry{
	paperwidth=2.75in,
	paperheight=4.25in,
	top=0.2in,
	bottom=0.5in,
	left=0.2in,
	right=0.2in
}

\usepackage{tikz}
\usetikzlibrary{shapes.geometric, arrows.meta, positioning}

\input{src/how-to-consensus-zine/flow-chart-only.tex}%

\setcounter{secnumdepth}{0}

\begin{document}

	\hspace*{-0.5cm}
		\consensusflowchart{0.6}{\footnotesize} 
	\newpage% Smaller version
	 
	\section{Consensus Process}
	
	\textbf{Discuss}: The item is discussed with the goal of identifying opinions and information on the topic at hand.
	
	\textbf{Proposal}: Based on the discussion a decision proposal is presented
	
	\textbf{Consensus Test}: The facilitator calls for consensus on the proposal. Each member must actively state whether they agree/consent, stand aside, or object,
	
	\textbf{Stand Aside}: Member does not support a proposal, but does not block
	
	\textbf{Block}: Member blocks on moral or ethical grounds and the proposal fails
	
	\newpage
	
	\section{Why Consensus}
	Some resources can't or shouldn't be managed by a central authority to preserve those resources for the good of those who benefit, present and future, from those resources. When central authorities aren't reliable or effective enough people can collectively come to decisions, where every individual has their ideas and interests protected. It's a proven method that preserves expertise while allowing all perspectives to be impactful. It builds group trust and awareness
	
	\newpage
	\section{Roles}
	Roles: Essential for      Efficiency
	
	Essential Roles:        
	
	Facilitator: Keep the group on topic on time and within the rules        
	
	Timekeeper: Make sure no one rambles and keep on schedule         
	
	 Note Taker: Takes Notes     
	 
	 DLC Roles:   
	 
	 Empath: keep the emotional climate rational diffuse potential emotional conflicts    
	 
	 Devils Advocate: you know this one    Greeter: greet newcomers, inform them of what's happened
	
	\newpage
	\section{Things go right when...}
	\begin{enumerate}
	\item meeting more frequently
	\item use direct action tactics- paid staff are avoided- networked with other  consensus-based groups
	\item members monitor their own/others’ domineering behavior
	\item members reflected collectively on the distribution of power
	\item as many decisions as possible are left up to each person
	\item all decisions are treated as provisional
	\end{enumerate}
	\newpage
	\section{Things go wrong when...}
	\begin{enumerate}
	\item one voice is heard more than others
	\item complacency in a few leading decisions
	\end{enumerate}
	\newpage
	
	Back\par Cover 
	\thispagestyle{empty} 
	
	\newpage
	

	% Add background image
	\AddToShipoutPictureBG*{%
		\AtPageLowerLeft{%
			\includegraphics[width=\paperwidth,height=\paperheight]{src/how-to-consensus-zine/border-edited.png}%
		}%
	}
	

	\begin{center}
		{\Huge\textbf{Intro to}}\\[0.3cm]
		{\Huge\textbf{Achieving}}\\[0.3cm]
		{\Huge\textbf{Consensus}}\\[1cm]
		
		{\Large Consensus Decision Making (CDM)}\\[0.2cm]
		{\Large Intro and Reference}\\[1cm]
		
		\emph{``The commons are those things\\
			that we all own together,\\
			that are neither privately owned,\\
			nor managed by the government.''}
			
		\vspace{1cm}
		
	\end{center}
		

	 \thispagestyle{empty} 
	 
\end{document}
