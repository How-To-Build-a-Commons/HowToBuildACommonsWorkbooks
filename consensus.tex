	\section{What is Consensus Process?}

Consensus is a process where everyone should be able to weigh in equally on a decision, and no one should be bound by a decision they detest. This boils doing in practice to: Everyone who feels they have (something relevant to the commons) to say about a proposal ought to have their perspectives carefully considered. When making decision for the commons, every member should have the right to participate equally.

"Everyone who has strong concerns or objections should have those concerns or objections taken into account and, (if allowed by the commons structure), addressed in the final form of the proposal.

Anyone who feels a proposal violates a(n organizing principal of the common) should have the opportunity to veto (“block”) that proposal.

No one should be forced to go along with a decision to which they did not assent. Likewise, meetings should not be held when there is not a problem to be solved. 

A consensus meeting needs some structure to function. When running a meeting there are some roles that need to be filled. Someone needs to act as the facilitator. A facilitator keeps the notes on the meeting and keeps the current version of the proposal. Their role is to ensure that the proposals and objections are recorded accurately. A time keeper is needed to ensure that the agreed upon time limit for speaking is adhered to. Someone should also be keeping track of hands raised to keep the order of the discussion. If just one person is available, they could fulfill these roles themselves, but if the task is shared, then that is preferable. 

\begin{enumerate}[]
	\item (A member of the common) makes a proposal for a certain course of action
	\item The facilitator asks for clarifying questions to make sure everyone understands precisely what is being proposed
	\item the facilitator asks for concerns
	\begin{enumerate}
		\item during the discussion those with concerns may suggest friendly amendments to the proposal to address the concern, which the person originally bringing the proposal may or may not adopt
		\item there may or may not be a temperature check about the proposal, an amendment, or the seriousness of a concern
		\item in the course of this the proposal might be scotched, reformulated, combined with other proposals, broken into pieces, or tabled for later discussion.
	\end{enumerate}
	\item the facilitator checks for consensus by:
	\begin{enumerate}
		\item asking if there are any stand-asides. By standing aside one is saying “I don’t like this idea, and wouldn’t take part in the action, but I’m not willing to stop others from doing so”. It is always important to allow all those who stand aside to have a chance to explain why they are doing so.
		
		\item asking if there are any blocks. A block is not a “no” vote. It is much more like a veto. Perhaps the best way to think of it is that it allows anyone in the group to temporarily don the robes of a Supreme Court justice and strike down a piece of legislation they consider unconstitutional; or, in this casein violation of the fundamental principles of unity or purpose of being of the group. Note: I should note that the usual language in Occupy Wall Street is that a block has to be based on a “moral, ethical, or safety concern that’s so strong you’d consider leaving the movement were the proposal to go forward”.,
		
	\end{enumerate}
\end{enumerate}
" \footnote{Democracy Project - Chapter 2 }