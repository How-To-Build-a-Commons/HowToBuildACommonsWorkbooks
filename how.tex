	\section{HOW-TO Build A Common}


Lets imagine we would actually like to put these ideas into practice. 

\begin{enumerate}
	
	\item The first task is to gather together a group of your community to discuss what to organize. Using the stated principals in this pamphlet is an easy way to begin discussion. A group can change any and all frameworks laid out here. The important part is to agree on principals of decision making and intent as the starting point. Everything starts by talking to the people around you. 
	
	\item Define a piece of the physical world with clearly defined boundaries. A clearly defined boundary will be some piece or pieces of the physical world that can be assigned as property by the legal system in which you reside. The group that is gathering can use the consensus process to determine what piece of the physical world they wish to manage. If it is a purchase of a property to manage, discuss the monthly input committed by each member, and compare that to the available real estate and financing terms. A trust is a legal structure that can be used with a charter of assigned property and access rights to bridge cooperative decision making internally with the legal structure external to the cooperative. 
	
	\item Establish problem solving strategies and conflict resolution forums. This can simply be that you agree to call a consensus meeting whenever there is a commons related decision that needs to be made. Ostrums book contains many different forms of problem solving systems if you need more examples. 
	
\end{enumerate}
