\section{Why Build a Common?}

From the Yana Ludwig reading:


\subsection{Spiritual or Religious}
Likely the oldest form of intentional community, these groups are organized to support religious practices or spiritual beliefs while minimizing material suffering. Modern versions of this type of effort include ashrams, nunneries, and monasteries. However, there are other spiritual communities that defy neat labels.
\subsection{Cultural Preservation}
When groups feel as though the world is changing quickly or in a way that doesn't match with their values, they may form a community focused on preserving treasured practices of the past. Amish, Indigenous, and back to the land groups all have in common a focus on promoting traditional skills and knowledge despite different underlying values. 
\subsection{Social Experimentation}
Some groups are focused on the evolution of a specific social structure or dynamic such as through justice and anti-oppression work or the development of more authentic interpersonal relationships. 
\subsection{Service-based}
Elements of service are common among intentional communities. Some are organized around the idea of service, while others have programs that serve their broader communities. This can include organizations that work with homeless populations or even organizations that run bulk food purchasing clubs.
\subsection{Economic Security}
In high cost of living areas, communities formed around economic security can be particularly effective at allowing creative and other groups to remain in place as costs rise. Communities can use tools such as income pooling, cost splitting, or cooperatively run business ventures to support economic access.
\subsection{Identity-based Safe Havens}
"From war resistance to LGBTQIA+ enclaves to liberatory organizing for Black power, identity-based safe havens are another reason people do community." It facilitates marginalized communities organize and normalize their identities withing the group. They allow individuals to grow, a place to belong not be set apart.
\subsection{Lifestyle and Comfort Enhancement}
Having a safe and comfortable neighborhood, improving access to amenities for your family, and using communal mechanisms to improve one's overall quality of life at reduced cost are another set of drivers for communities to form. Many Cohousing organizations in the US fall into this category.
\subsection{Ecological Sustainability}
Communities dedicated to ecological sustainability area a relatively recent phenomenon. This type could include anywhere from modern ecovillage groups and range to all those that view humans and the land as being in partnership with one another. 