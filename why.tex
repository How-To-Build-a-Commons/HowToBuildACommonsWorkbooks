\section{Why Build a Common?}

From the Yana Ludwig reading:


\subsection{Spiritual or Religious}
\subsection{Cultural Preservation}
\subsection{Social Experimentation}
\subsection{Service-based}
Elements of service are common among intentional communities. Some are organized around the idea of service, while others have programs that serve the broader communities. This can include organizations that work with homeless populations or even organizations that run bulk food purchasing clubs.
\subsection{Economic Security}
In high cost of living areas, communities formed around economic security can be particularly effective at allowing creative and other groups to remain in place as costs rise. Communities can use tools such as income pooling, cost splitting, or cooperatively run business ventures to support economic access.
\subsection{Identity-based Safe Havens}
"From war resistance to LGBTQIA+ enclaves to liberatory organizing for Black power, identity-based safe havens are another reason people do community." It facilitates marginalized communities organize and normalize their identities withing the group. They allow individuals to grow, a place to belong not be set apart.
\subsection{Lifestyle and Comfort Enhancement}
\subsection{Ecological Sustainability}