	\section{What is a Common}

\vspace{0.2cm}

As we live in a world that is entirely governed by markets, I find myself wondering, can we make some of the world different? It turns out that there is a type of managing resources that is different than the State ownership, and different than private ownership, it's commons ownership. 

A commons is some piece of the physical world that is managed by a community for the benefit of that community. A commons can be a shared resource that community members share time using in turn. A commons can also manage a consumed resource that is allocated to community members according to the agreements of the community. A community member that receives something from a commons is an "appropriator".

A commons can be managed in a very large variety of ways. Members create the rules that are used to govern their own commons. Some commons are more and less stable over time, depending on what rules the community chooses to adopt. 


Lets review some design principals used by people across the world for stable commons. These principals were derived by Ostrum from a datasets of commons' that have been operating for at least 200 years. 

"
\begin{enumerate}
	
	\item {Clearly defined boundaries}
	
	Individuals or households who have rights to withdraw resource units from the commons must be clearly defined, as must the boundaries of the commons itself.
	
	
	\item{ Congruence between appropriation and provision rules and local conditions}
	
	Appropriation rules restricting time, place, technology, and/or quantity of resource units are related to local conditions and to provision rules requiring labor, material, and/or money.
	
	
	\item{Collective-choice arrangements}
	
	community members can participate in modifying the operational rules.
	
	
	\item{Monitoring}
	
	Monitors, who actively audit commons conditions and appropriator behavior, are accountable to the appropriators or are the appropriators.
	
	
	\item{Graduated sanctions}
	
	Appropriators who violate operational rules are likely to be assessed graduated sanctions (depending on the seriousness and context of the offense) by other appropriators, by officials accountable to these appropriators, or by both. See the reading for many varied examples. 
	
	
	\item{Conflict-resolution mechanisms}
	
	Appropriators and their officials have rapid access to low-cost local arenas to resolve conflicts among appropriators or between appropriators and officials.
	
	
	
	\item{Minimal recognition of rights to organize}
	
	The rights of appropriators to devise their own institutions are not challenged by external governmental authorities. Essentially property rights need to exist and be assigned to the commons itself. 
	
	For commons that are parts of larger systems:
	
	\item{Nested enterprises}
	
	Appropriation, provision, monitoring, enforcement, conflict resolution, and governance activities are organized in multiple layers of nested enterprises.
	
	"\footnote{"Governing the Commons p90"}
	
\end{enumerate}
\vspace{0.2cm}
Always remember that building a common is a step-by-step process. You will start with whatever resources and rules you have, and add to the framework over time. The whole idea is to have the rules reflect the needs of the members of the commons. We are all screw-ups some times, and that is ok! We can always evolve our systems over time and bring them in line with our shared values. 



